\documentclass[size=a4, parskip=half, titlepage=false, toc=flat, toc=bib, 12pt]{scrartcl}

\setuptoc{toc}{leveldown}

% Ajuste de las líneas y párrafos
\linespread{1.2}
\setlength{\parindent}{0pt}
\setlength{\parskip}{12pt}

% Español
\usepackage[spanish, es-tabla]{babel}

% Matemáticas
\usepackage{amsmath}
\usepackage{amsthm}

% Fuentes
\usepackage{newpxtext,newpxmath}
\usepackage[scale=.9]{FiraMono}
\usepackage{FiraSans}
\usepackage[T1]{fontenc}

\defaultfontfeatures{Ligatures=TeX,Numbers=Lining}
\usepackage[activate={true,nocompatibility},final,tracking=true,factor=1100,stretch=10,shrink=10]{microtype}
\SetTracking{encoding={*}, shape=sc}{0}

\usepackage{graphicx}
\usepackage{float}

% Mejores tablas
\usepackage{booktabs}

\usepackage{adjustbox}

% COLORES

\usepackage{xcolor}

\definecolor{verde}{HTML}{007D51}
\definecolor{esmeralda}{HTML}{045D56}
\definecolor{salmon}{HTML}{FF6859}
\definecolor{amarillo}{HTML}{FFAC12}
\definecolor{morado}{HTML}{A932FF}
\definecolor{azul}{HTML}{0082FB}
\definecolor{error}{HTML}{b00020}

% ENTORNOS
\usepackage[skins, listings, theorems]{tcolorbox}

\newtcolorbox{recuerda}{
  enhanced,
%  sharp corners,
  frame hidden,
  colback=black!10,
	lefttitle=0pt,
  coltitle=black,
  fonttitle=\bfseries\sffamily\scshape,
  titlerule=0.8mm,
  titlerule style=black,
  title=\raisebox{-0.6ex}{\small RECUERDA}
}

\newtcolorbox{nota}{
  enhanced,
%  sharp corners,
  frame hidden,
  colback=black!10,
	lefttitle=0pt,
  coltitle=black,
  fonttitle=\bfseries\sffamily\scshape,
  titlerule=0.8mm,
  titlerule style=black,
  title=\raisebox{-0.6ex}{\small NOTA}
}

\newtcolorbox{error}{
  enhanced,
%  sharp corners,
  frame hidden,
  colback=error!10,
	lefttitle=0pt,
  coltitle=error,
  fonttitle=\bfseries\sffamily\scshape,
  titlerule=0.8mm,
  titlerule style=error,
  title=\raisebox{-0.6ex}{\small ERROR}
}

\newtcblisting{shell}{
  enhanced,
  colback=black!10,
  colupper=black,
  frame hidden,
  opacityback=0,
  coltitle=black,
  fonttitle=\bfseries\sffamily\scshape,
  %titlerule=0.8mm,
  %titlerule style=black,
  %title=Consola,
  listing only,
  listing options={
    style=tcblatex,
    language=sh,
    breaklines=true,
    postbreak=\mbox{\textcolor{black}{$\hookrightarrow$}\space},
    emph={jmml@UbuntuServer, jmml@CentOS},
    emphstyle={\bfseries},
  },
}

\newtcbtheorem[number within=section]{teor}{\small TEOREMA}{
  enhanced,
  sharp corners,
  frame hidden,
  colback=white,
  coltitle=black,
  fonttitle=\bfseries\sffamily,
  %separator sign=\raisebox{-0.65ex}{\Large\MI\symbol{58828}},
  description font=\itshape
}{teor}

\newtcbtheorem[number within=section]{prop}{\small PROPOSICIÓN}{
  enhanced,
  sharp corners,
  frame hidden,
  colback=white,
  coltitle=black,
  fonttitle=\bfseries\sffamily,
  %separator sign=\raisebox{-0.65ex}{\Large\MI\symbol{58828}},
  description font=\itshape
}{prop}

\newtcbtheorem[number within=section]{cor}{\small COROLARIO}{
  enhanced,
  sharp corners,
  frame hidden,
  colback=white,
  coltitle=black,
  fonttitle=\bfseries\sffamily,
  %separator sign=\raisebox{-0.65ex}{\Large\MI\symbol{58828}},
  description font=\itshape
}{cor}

\newtcbtheorem[number within=section]{defi}{\small DEFINICIÓN}{
  enhanced,
  sharp corners,
  frame hidden,
  colback=white,
  coltitle=black,
  fonttitle=\bfseries\sffamily,
  %separator sign=\raisebox{-0.65ex}{\Large\MI\symbol{58828}},
  description font=\itshape
}{defi}

\newtcbtheorem{ejer}{\small EJERCICIO}{
  enhanced,
  sharp corners,
  frame hidden,
  left=0mm,
  right=0mm,
  colback=white,
  coltitle=black,
  fonttitle=\bfseries\sffamily,
  %separator sign=\raisebox{-0.65ex}{\Large\MI\symbol{58828}},
  description font=\itshape,
  nameref/.style={},
}{ejer}


% CÓDIGO
\usepackage{listings}

% CABECERAS
\pagestyle{headings}
\setkomafont{pageheadfoot}{\normalfont\normalcolor\sffamily\small}
\setkomafont{pagenumber}{\normalfont\sffamily}

% ALGORITMOS
\usepackage[vlined,linesnumbered]{algorithm2e}

% Formato de los pies de figura
\setkomafont{captionlabel}{\scshape}
\SetAlCapFnt{\normalfont\scshape}
\SetAlgorithmName{Algoritmo}{Algoritmo}{Lista de algoritmos}

% BIBLIOGRAFÍA
%\usepackage[sorting=none]{biblatex}
%\addbibresource{bibliografia.bib}


\begin{document}

\renewcommand{\proofname}{\normalfont\sffamily\bfseries\small DEMOSTRACIÓN}

\title{Memoria práctica 1}
\subject{Técnicas de los Sistemas Inteligentes}
\author{Johanna Capote Robayna \\ Guillermo Galindo Ortuño}
\publishers{\vspace{2cm}\includegraphics[height=2.5cm]{UGR}\vspace{1cm}}
\maketitle


\newpage

\section{Descripción general de la solución}

En esta práctica se nos pedía programar un controlador del entorno GVG\_AI capaz
de guiar a un avatar que fuera capaz de pasar los distintos niveles del juego
\texit{Boulder Dash}. Este avatar aparte de un comportamiento deliberativo basado
en el algoritmo A* debía ser capaz de esquivar a los enemigos mediante un
comportamiento reactivo.

Esta memoria se dicidirá en tres partes, una descripción general de la solución,
otra del comportammiento reactiva y la última explicara el comportamiento deliverativo.

El esquema general del comportamiento del avatar es el siguiente:
\begin{itemize}
\item La estrategia de búsqueda consiste en ordenar las gemas según la longitud del camino que el algoritmo A*
proporciona para llegar a cada una de ellas, con esto se busca optimizar el
tiempo, ya que no siempre la gema más cercana al avatar tiene el camino más corto.

\item El avatar comienza con un comportamiento deliberativo, donde va recolectando
las gemas en el orden proporcionado, cuando ya obtiene las gemas necesarias para
pasar de nivel, llama de nuevo al algoritmo A* para encontrar el camino más corto
hacia el portal y así superar el nivel.

\item En caso de que el avatar detecte que pueda morir en el siguiente movimiento
se activa el comportamiento reactivo, con el que se busca que el agente
sepa reaccionar a situaciones inesperadas.

\item La heurística empleada es la distancia Manhatam, se ha decidido no cambiar
la heuristica puesto que es admisible, lo cual hace que A*
 encuentre siempre un camino óptimo a la gema que buscamos en caso de que exista.

\item El aspecto más relevante del controlador, es su capacidad de reaccionar
a cualquier imprevisto, ya que debido a la características de los mapas
proporcionados era uno de los principales problemas a resolver. El hecho de que
el mapa pueda cambiar con el desprendimiento de rocas y la aprición de enemigos
que puedan provocar la muerte del avatar, hacen que este sea el principal objetivo
a resolver.

\item Por último si pasa más de 4 turnos sin encontrar ningún camino hacia alguna
gema, el avatar cambia de objetivo y se dirige a la piedra movible más cercana con
la intención de que se mueva y abrá un nuevo camino hacia alguna gema.
\end{itemize}

\newpage

\section{Comportamiento reactivo}
Como se comento anteriormente, el comportamiento reactivo es el que más importancia
se le ha dado en esta práctica ya que era el principal motivo de muerte del avatar.
Este se basa en una única función \textit{esPeligrosa}, en esta función se gestiona
todo el comportamiento reactivo.

En primer lugar, se crea un \texit{array} de posibles acciones, de las cuales
se devolverá alguna. A continuación se recrea el movimiento que el comportamiento
deliberativo planeaba hacer, y se comprueba si el agente muere al realizar dicha acción.
En caso de no morir, se comprueba que el movimiento no le lleve a una muerte segura,
es decir que en el hipotético caso de hacer el movimiento halla al menos un movimiento que
pueda hacer en el siguiente turno sin morir. Si una vez recreado el movimiento hay algún
movimiento en el que no muera, se devuelve la acción planeada, de no ser así se borra de
la lista de posibles acciones y se busca una acción que no lleve al agente a una muerte segura.

En la segunda parte de la función, una vez descartado el movimiento que devolvió el comportamiento
deliberativo puesto que este conducía a una muerte segura, pasamos a buscar otra acción. Para ello
recorremos el \texit{array} de posibles acciones repitiendo el proceso anterior, recreamos cada una
de las posibles acciones por orden y devolvemos aquella que no conduce a una muerte segura. En caso
de no encontrar ninguna acción, se asume que el avatar va a morir y se devuelve la acción \textit{NIL}.

\begin{algorithm}[H]
 \KwData{StateObservation stateObs, Types.ACTIONS siguienteaccion}
 \KwResult{Types.ACTIONS siguienteaccion}
 posibles\_acciones = Types.ACTIONS.values()\;
 peligro = false\;
 aux\_stateobs = stateObs.copy()\;
 aux\_sateobs.advance(siguienteaccion)\;

 peligro = !aux\_stateobs.isAvatarAlive()\;
 \If{!peligro}{
    muere\_siempre = true\;
    indice = 0\;
    \Repeat{ aux\_stateobs.isAvatarAlive() \textbf{or} ind >= Types.ACTIONS.Values().length}{
      accion\_futura = Types.ACTIONS.values()[indice]\;
      aux\_stateobs = stateObs.copy()\;
      aux\_stateobs.advance(siguienteaccion)\;
      aux\_stateobs.advance(accion\_futura)\;
      indice ++\;
    }
    \If{aux\_stateobs.isAvatarAlive()}{muere\_siempre = false\;}
    \If{!muere\_siempre}{\Return siguiente\_accion \;}
    \Else{posibles\_acciones.remove(siguienteaccion) \;}
 }
 \For{accion\_candidata \textbf{in} posibles\_acciones}{
    \If{aux\_stateobs.isAvatarAlive()}{
    muere\_siempre = true\;
    indice = 0\;
    \Repeat{ aux\_stateobs.isAvatarAlive() \textbf{or} ind >= Types.ACTIONS.Values().length}{
      accion\_futura = Types.ACTIONS.values()[indice]\;
      aux\_stateobs = stateObs.copy()\;
      aux\_stateobs.advance(siguienteaccion)\;
      aux\_stateobs.advance(accion\_futura)\;
      indice ++\;
    }
    \If{aux\_stateobs.isAvatarAlive()}{muere\_siempre = false\;}
    \If{!muere\_siempre}{\Return accion\_candidata \;}
    }
 }
 \Return Types.ACTIONS.ACTION\_NIL\;
 \caption{función: \textit{esPeligrosa}}
\end{algorithm}

\newpage

\section{Comportamiento deliberativo}
El comportamiento deliberativo del agente es bastante sencillo, este se reduce
a llamar al algoritmo A* para buscar un camino hacia cada gema. Estas gemas,
como se comentó anteriormente, están ordenadas por la longitud del camino.
En caso de no encontrar un camino (A* devuelve \texit{null}) se elige como
siguiente acción \textit{NIL} y se llama a la función \textit{esPeligrosa}
para garantizar que no muera en el siguiente movimiento.

Por otro lado, si el avatar pasa más de dos movimientos devolviendo la acción
\textit{NIL}, el mapa se actualiza, por si ha habido algún desprendimiento de
una piedra abriendo un nuevo camino.

Por último, si el avatar pasa más de 4 movimientos en el que el algoritmo A*
devuelve \textit{null} el avatar cambia de objetivo y busca el primer objeto
movible del mapa, y se dirije a la casilla que se encuentra justo debajo. En
caso de no encontrar un camino hacia esta \textit{roca objetivo} cambia de
roca (\textbf{nRocas}).

\begin{algorithm}
\If{ticks\_sin\_caminos > 4}{
  posiciones\_rocas = stateObs.getMovablePositions()\;

  \For{roca \textbf{in} posiciones\_rocas}{
    pos\_debajo\_rocas.add( roca.position.x , roca.position.y +1) \;
  }

  \If{pos\_debajo\_rocas.size() > nRocas}{
    pos = pos\_debajo\_rocas.get(nRocas)\;

    path = pf.astar._findePath(avatar, pos)\;

    if{path == null}{ nRocas ++ \;}
  }
}

\end{algorithm}
%printbibliography


\end{document}
